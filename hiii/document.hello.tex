\documentclass[14pt,a4paper]{article}
\usepackage[utf8]{inputenc}
\usepackage[T1]{fontenc}
\title{PAPA KI PARI}
\author{Dr. Johnny Sins}
\begin{document}
	\maketitle
	\section{SIMPLE LINEAR EQUATION}
	\subsection{uncle ji pani pila dijiye}
	Linear Equations
	Linear equations are equations of the first order. The linear equations are defined for lines in the coordinate system. When the equation has a homogeneous variable of degree 1 (i.e. only one variable), then it is known as a linear equation in one variable. A linear equation can have more than one variable. If the linear equation has two variables, then it is called linear equations in two variables and so on. Some of the examples of linear equations are 2x – 3 = 0, 2y = 8, m + 1 = 0, x/2 = 3, x + y = 2, 3x – y + z = 3. In this article, we are going to discuss the definition of linear equations, standard form for linear equation in one variable, two variables, three variables and their examples with complete explanation.
	
	An equation is a mathematical statement, which has an equal sign (=) between the algebraic expression. Linear equations are the equations of degree 1. It is the equation for the straight line. The solutions of linear equations will generate values, which when substituted for the unknown values, make the equation true. In the case of one variable, there is only one solution. For example, the equation x + 2 = 0 has only one solution as x = -2. But in the case of the two-variable linear equation, the solutions are calculated as the Cartesian coordinates of a point of the Euclidean plane.
	
	In practice, we don't really try even possible line.Instead we use calculus to find the values of $b_{0}$ and $b_{1}$ that give the minimum sum of squared residuals. You don't need to memorize or use these equations,but here they are in case you are interested.
	
	
         	$ b_{1}= \frac{\Sigma_{i=1}^{n} (x_{i}-\overline{x})(y_{i}-\overline{y})}{(x_{i}-\overline{x})^2}$\\
         	
     Also,the best estimate of $\sigma^2$ is \\
     
     $ 8^2= \frac{\Sigma_{i=1}^{n}(y_{i} - \widehat{y}_{i}) ^2}{n-2} $
     
     Whenever we ask a computer to perform simple liner Both sides of the equation are supposed to be balanced for solving a linear equation. The equality sign denotes that the expressions on either side of the ‘equal to’ sign are equal. Since the equation is balanced, for solving it, certain mathematical operations are performed on both sides of the equation in a manner that does not affect the balance of the equation. Here is the example related to the linear equation in one variable.
 
    Here are the derivations of the coeffceint estimates.SSR indictes sum of sqyared residuals the quantity to minimize.
    
    $SSR = \Sigma_{i=1}^{n}(y_{i}-(\beta_{0}+\beta_{1}x_{i}))^2$\\
    
   $ =  \Sigma_{i=1}^{n}(y_{i}^2(\beta_{0}+\beta_{1}x{i})+\beta_{0}^2+2\beta_{0}\beta_{1}x_{i}+\beta_{1}^2x_{i}^2) $\\
   
   $\frac{\partial SSR}{\partial \beta_{0}} = \Sigma_{i=1}^{n}(-2y_{i} + 2\beta_{0} + 2\beta_{1}x_{i}) $\\


$0 = \Sigma_{i=1}^{n}(-y_{i}+\widehat{\beta_{0}}+\widehat{\beta_{1}}x_{i}) $\\


$0 = -n\overline{y} + n\widehat{\beta_{0}} + \widehat{\beta_{1}}n\overline{x}$\\

$\widehat{\beta_{0}} = \overline{y} - \widehat{\beta_{0}}\overline{x} $\\
  
  $\frac{\partial SSR}{\partial\beta_{1}} = \Sigma_{i=1}^{n}(-2x_{i}y_{i} + 2\beta_{0}x{i} + 2\beta_{1}x_{i}^2 )$ \\ 
 
 
 $0 = -\Sigma_{i=1}^{n}x_{i}y_{i}+ \widehat{\beta_{0}}\Sigma_{i=1}^{n}x_{i} + \widehat{\beta_{1}}\Sigma_{i=1}^{n}x_{i}^2$\\
 
 $0 = -\Sigma_{i=1}^{n}x_{i}y_{i}+ (\overline{y}-\widehat{\beta_{1}}\overline{x})\Sigma_{i=1}^{n}x_{i}+\widehat{\beta_{1}}\Sigma_{i=1}^{n}x_{i}^2$\\
 
 $\widehat{\beta_{1}} = \frac{\Sigma_{i=1}^{n}x_{i}(y_{i}-\overline{y})}{\Sigma_{i=1}^{n}x_{i}(x_{i}-\overline{x})}$
     
     A little algebra shows that the formula for $\widehat{\beta_{1}}$ is equivalent to the one  shown above because $c\Sigma_{i=1}^{n}(z_{i}-\overline{z}) = c*0 = 0$ for any constant c and variable z.
     
  
     
\end{document}